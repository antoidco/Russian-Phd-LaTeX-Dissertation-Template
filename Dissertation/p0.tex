\chapter{Литературный обзор}

\fixme{Из философии науки. Переделать обязательно, так как 90\% --- вода, никак не связанная с работой.}

Введение

Вопросы управления нелинейными волновыми процессами – относительно новая и недостаточно изученная область науки в силу существенно ограниченной применимости разрабатываемых методов и алгоритмов управления. Однако построение наиболее общих математических моделей управления такими процессами – очень перспективное направление для исследований. Наука о нелинейных волнах и колебаниях появилась относительно недавно и продолжает активно развиваться по сей день. Объекты ее исследования продолжают находить себе применение, открывать новые ранее недоступные технологические решения. В условиях такого бурного развития могут стать востребованы средства управления нелинейными волнами, и, как следствие, работающий математический аппарат, описывающий методы управления и границы их применимости. 

Управление нелинейными волновыми процессами непосредственно связано с исследованием природы этих процессов, в том числе, для анализа эффективности или применимости того или иного вида управления. В прошедшем столетии был разработан ряд общих методов исследования нелинейных волновых процессов. Они существенно отличаются от методов, относящихся к линейным задачам, для которых, например, существуют общие способы нахождения точных решений определяющих уравнений. Методы же анализа нелинейных волновых процессов существенно отличаются и традиционно классифицируются на методы поиска точных решений, построения асимптотических приближений и определения границ применимости данных методов к конкретной задаче. Таким образом, для работы в данной области исследования необходимо детально рассмотреть методологию исследования нелинейных волновых процессов, а также историю их изучения.

Также необходимо рассмотреть вопросы теории управления, устоявшиеся методы и алгоритмы управления линейными волновыми процессами, а также текущее состояние применения современных способов управления нелинейными системами и перспективы их дальнейшего развития. Разумеется, последнее непосредственно связано с практической применимостью методов управления, рассмотрению сред и материалов, в которых это управление может осуществляться. 

Таким образом, среди целей данной работы следует выделить вопросы истории и методологии исследования нелинейных волновых процессов, также необходимо уделить внимание истории развития методов управления линейными и нелинейными механическими системами, при этом рассмотреть вопросы о применимости разрабатываемого математического аппарата в ходе исследовательской работы, акцентируя внимание на методы исследования тех областей науки и техники, к которым он будет применяться в перспективе.

Глава 1. Нелинейные волновые процессы. История и методы исследования 

Ключевая особенность и историческая новизна описания нелинейных волновых процессов связана именно с «нелинейностью» рассматриваемых систем. В большинстве случаев, термин «нелинейность» подразумевает, дословно, «отсутствие линейности», то есть, например, отсутствие простейшего линейного соотношения в описании связи некоторых физических величин. Вместо него вводится более общая и более сложная, так называемая, нелинейная связь, которая может, в том числе, существенно изменить вид определяющих дифференциальных уравнений в частных производных, описывающих поведение рассматриваемой системы и выведенных из физических законов с учетом этой введенной и более общей с точки зрения физики нелинейной связи. Их вид при этом становится более сложным, в математическом смысле говорят, что уравнения теряют свойства линейности. Нахождение точных решений нелинейных дифференциальных уравнений зачастую становится гораздо более сложной задачей, так как соответствующие методы поиска решений, работающие для линейных уравнений, становятся неприменимы. Для нелинейных уравнений общих методов поиска точных решений не существует. Таким образом, математический смысл термина «нелинейность» подразумевает усложнение уравнений математической модели физического явления, приводящей к нарушению свойств их линейности, а физический – стремление к получению более общей модели явления с целью описания и исследования его новых свойств, дающих, соответственно, более общее и правильное представление о мире.

В науке долгое время «правильными» считались именно линейные уравнения, считалось, что именно они описывают «чистую» физику и являются фундаментальными. Нелинейные постановки задач при этом подразумевались скорее некими частными и редкими случаями, за которыми не стоят фундаментальные физические принципы. В особенности, это касалось уравнений, описывающих волновые процессы. В том числе, электромагнитные волны или акустические волны (волны на поверхности воды, механические волны в упругой среде). Толчком к развитию и исследованию нелинейных волновых процессов стало исследование уединенных волн на поверхности воды.

Впервые внимание уединенным волнам на поверхности воды уделил талантливый конструктор и кораблестроитель Джон Скотт Рассел в 1834 году и изложил свои наблюдения в «Докладе о волнах» [27]. Он заметил необычную волну, движущуюся вдоль канала с одинаковой скоростью в виде «водяного холма», форма которого не менялась и сохраняла устойчивость в ходе движения. Наблюдаемая волна двигалась со скоростью примерно 10-15 километров в час и сохраняла свою колоколообразную форму на протяжении нескольких километров. Примечательно, что с данным явлением наверняка сталкивались многие другие люди, но именно Джон Скотт Рассел смог увидеть в нем эту самую необычность с точки зрения физики того времени, а написанный им и опубликованный «доклад о волнах» стал отправной точкой в изучении нелинейных волн. 

Сначала Рассел был подвергнут критике авторитетами того времени в области науки. К тому времени гидродинамика уже стала математически систематизированной наукой, и все ее задачи были сведены к теории дифференциальных уравнений в частных производных. При этом формула для скорости волны, которую опубликовал Рассел в своем докладе, не получалась из теории длинных волн на мелкой воде. К тому же уравнения этой теории не допускали сохранение формы волны такой большой амплитуды. Первыми, кто уделили внимание докладу Рассела, были Джордж Биддель Эйри и Джордж Габриель Стокс. В одной из своих работ [16] хорошо известный Эйри подверг категорической критике работу Рассела, назвав его наблюдения ошибочными, и, тем самым, привлек существенное внимание к данному вопросу. Стокс же подошел к вопросу уже более внимательно и доказал, что наблюдения Рассела не соответствуют теории даже в рамках пренебрежимо малой вязкости жидкости [28], таким образом показав, что даже в отсутствие диссипативных сил волна все равно не могла сохранить свою форму и должна была распасться. Таким образом, он также указал на то, что наблюдения Рассела не должны соответствовать действительности.

Немного позднее в семидесятых годах того же столетия ученику Стокса, лорду Рэлею [26], а также ученому Жозефу Валентину де Буссинеску [17] независимо друг от друга удалось найти приближенное математическое описание скорости и формы уединенной волны на поверхности мелкой воды. В то же время появились экспериментальные работы, доказывающие существование уединенных волн и подтверждающие наблюдения Рассела. Однако споры о существовании и описании уединенных волн продолжались, критика Эйри и Стокса была далеко небезосновательной. Казалось, что науке о волнах и колебаниях требовались новые подходы для ее расширения и обобщения для описания все более очевидно существующих объектов – уединенных волн.

Впервые подобное обобщение, внесшее наибольшую ясность в данном вопросе, провели [20] голландские ученые Дидерик Иоханнес Кортевег и его ученик Густав де Вриз. Они получили довольно простое уравнение для волн на поверхности воды, допускавшее в виде точного решения уединенную волну наиболее точно подходившую под описание Рассела, обобщив метод Релэя. Согласно полученному ими уравнению, волны на поверхности воды имеют синусоидальную форму только при очень низких амплитудах колебаний, в то время как при увеличении амплитуды, синусоида приобретает форму существенно удаленных друг от друга «холмов», которые и являются уединенными волнами. Таким образом, средствами их обобщения был получен предельный переход от их уравнения к классическому виду, который теперь предполагался справедливым только для волн малой амплитуды.
Позднее уравнение, выведенное голландскими учеными, получило имя его первооткрывателей и сейчас известно, как уравнение Кортевега-де Вриза или КдВ-уравнение. Оно нашло себе применение не только для описания волн на мелкой воде, но и в других областях физики, так как схожий обобщающий подход распространяется и на другие волновые процессы. С математической же точки зрения с этого уравнения начинается история солитона – так была в последствии названа структурно устойчивая уединённая волна, распространяющаяся в нелинейной среде. 

Тем не менее, исторически важность подхода и уравнения Кортевега-де Вриза для науки была понята не сразу. Скорее всего, это связано с тем, что область применимости нелинейных волн также оставалась для многих ученых непонятной. Так, для солитонов нарушались свойства линейности – например, принцип суперпозиции. Последний позволяет пользоваться сложением волн, объясняет явления дифракции и интерференции, им пользуется оптическая теория и на нем основаны принципы работы радиосвязи и телевидения. Все это вытекает из линейности описывающих физические системы уравнений. Эти факторы, а также отсутствие особой практической необходимости в локализованных волнах и проявление их только в особых случаях гидродинамике привели к серьезному перерыву в развитии науки о нелинейных волнах. 

Однако уже к середине двадцатого столетия постепенно начали появляться новые науки – нелинейная оптика, акустика, электродинамика, так как появились новые экспериментальные возможности исследования, а линейная теория подошла к своему пределу, и новые эффекты и явления можно было объяснить лишь в новом нелинейном рассмотрении. Динамика волн на мелкой воде перестала быть единственным оплотом нелинейной науки, а ее уравнения начали находить себе применение в совершенно других областях и доказали свою фундаментальность. Появились новые методы вывода уравнений, поиска их точных решений исследований, построения асимптотических приближений, аппараты для определения границ применимости данных методов. Родилась совершенно новая, нелинейная наука, призванная обобщить, уточнить и усложнить традиционную классическую.

Одно из явлений, вдохнувших новую жизнь в нелинейную науку, было связано с необычным поведением солитонов. Дело в том, что при рассмотрении системы солитонов – например, двух солитонов, движущихся с разной скоростью, при этом один догоняет другой, – их взаимодействие не разрушает систему. Более того, в приведенном случае солитоны после момента взаимодействия продолжат двигаться с изначальной скоростью и сохранят свою форму. Таким образом, можно обратить внимание на удивительное сходство солитонов с частицами. Впервые на это подобие уединенных волн частицам обратили внимание в ходе численных исследований американские ученые Норман Забаски и Мартин Дэвид Крускал уже в 1965 [29] году. Они и дали такое имя уединенной волне — солитон (уединенная волна – solitary wave), по аналогии с частицей (электрон — electron, позитрон — positron, мюон — muon и т.д.).
Очередной пример нелинейной локализованной волны, который привел к целому ряду исследований в области современной науки о нелинейных волнах – солитон Френкеля Конторовой, нашедший свое ряд применений при описании атомной модели движущейся дислокации [5]. Солитон Френкеля-Конторовой – особый дефект в кристаллической решетке твердого тела. Математическое описание таких объектов и точные решения уравнений показали, что движущиеся дислокации в кристалле взаимодействуют как солитоны, то есть они сохраняют свою форму после взаимных соударений. Помимо этого, в теории Френкеля-Конторовой выделяется два типа механических дислокаций: дислокация разрежения и дислокация сгущения, что механически соответствует «дырке» в решетке и «сгущению». В литературе они также называются дислокацией и антидислокацией. Одноименные дислокации в решетке отталкиваются друг от друга, а разноименные притягиваются друг к другу. При взаимодействии «дислокационного» и «антидислокационного» солитонов они меняются ролями, то есть дислокационнаый становится антидислокационным и наоборот. Здесь в очередной раз напрашивается аналогия с частицами, а именно каждой из дислокаций хочется сопоставить положительный или отрицательный заряды. При этом противоположные волны, как и частицы с античастицами могут попарно рождаться и исчезать, сохраняя «суммарный заряд» системы. Таким образом, довольно простая физико-математическая модель создает огромное пространство для размышлений и исследований в мире противоположно «заряженных» солитонов, в котором могут возникать задачи управления такими необычными объектами.

Необходимо также заметить, что модель движущейся дислокации описывается уравнением синус-Гордона, на примере которого был проведен ряд исследований в области управления с обратной связью с применением метода скоростного градиента в работах по теме исследования [22], [23], [24], [25].

Возникшая необходимость в решении задач нового типа, нелинейных волновых задач, требовала к ним нового методологического подхода. Анализ линейных задач часто использовал принцип суперпозиции, то есть сложения волн. Этот принцип сложения колебаний в свое время позволил разработать настолько эффективные методы решения, что обходиться без него было очень тяжело. Это привело к появлению до сих пор активно используемого метода линеаризации [21] уравнений. Метод заключается в том, чтобы «линеаризовать» задачу, то есть свести ее к такой, в которой в первом приближении выполняется принцип суперпозиции. Общих же подходов к решению нелинейных задач нет, поэтому и нет общей теории солитонов или нелинейных волн. Каждая новая задача по-своему уникальна, что делает эту науку очень богатой и требует от исследователя особого творческого подхода.
Таким образом, нелинейные волны и колебания играют очень важную роль в современной науке. В самом деле, все реальные динамические системы нелинейны. Их можно считать линейными лишь в случае рассмотрения в них волн достаточной малости. Рассмотренные уединенные волны — солитоны, а также другие типы локализованных волн, такие как кинки и бризеры, – с практической точки зрения интересны тем, например, что в средах особого типа они могут быть носителями информации о наличии дефектов в материале. Одна из практических задач, связанных с нелинейными волновыми процессами – генерация локализованных волн нужной формы и скорости. С теоретической точки зрения эта задача — задача теории управления, так как необходимо некоторым образом совершить воздействие на систему, чтобы изменить ее состояние (например, возбудить в ней солитон) до необходимого с определенной точностью. При этом, при выборе управления несомненно следует учитывать особенности нелинейной механической системы, в которой контролируется поведение нелинейной волны. Для этого необходимо иметь ясное представление и изучить природу нелинейных волновых процессов, что и было проделано в этой части работы. Кроме того, нужно заметить, что постановка такой задачи управления, вообще говоря, чисто математическая и учитывает свойства системы исходя из вида описывающих ее уравнений, а значит может найти себе применение и в других самых различных областях науки, где они могут встретиться.

Глава 2. Методы управления волновыми процессами

Родоначальником непосредственно «математической теории управления» можно считать Александра Михайловича Ляпунова — автора классической теории устойчивости движения (1892)[Евланов Л. Г., Самонастраивающаяся система с поиском градиента методом вспомогательного оператора. Изв. АН СССР, ОТН, «Техническая кибернетика», 1963, № 1.].
.....
Наука о нелинейных волнах и колебаниях намного моложе теории управления. Область применения последней может относиться не только к физике, ее математические аппараты могут находить себе применение и даже активней использоваться в экономике, социологии, политологии, правоведении, программировании, роботостроении и других областях. Ее применение сводится к поиску и обоснованию метода, алгоритма или способа привести некий процесс, систему или модель к целевому состоянию – цели управления. Такая потребность напрямую связана с любой деятельностью человека, поэтому она неявно существовала всегда, но была систематизирована и получила наибольший вектор развития к концу ХХ века. Тем не менее, нас интересует лишь область ее применения к ограниченному числу объектов исследования – а именно к волновым и колебательным процессам, а также некоторым другим нелинейным системам.
Методы управления колебательными и волновыми процессами начали активно развиваться в конце двадцатого столетия. Их теория брала за основу опыт управления классическими динамическими системами. 
Управление нелинейными системами может быть построено на основе методов теории оптимального управления [2]. Они могут учитывать определенные ограничения, наложенные на средства управления, а также и на фазовые координаты, что значительно усложняет задачу. Такой подход позволяет приводить рассматриваемую физическую систему к цели управления оптимальным способом, то есть, минимизируя некоторый введенный критерий, например, время. К сожалению, построение алгоритма управления средствами теории оптимального управления – существенно сложная задача, которая зачастую не имеет аналитических решений, особенно если говорить о сильно нелинейных системах.
Наряду с теорией оптимального управления существуют другие общие теории, основанные на различных вариациях метода управления системами переменной структуры [6], которые позволяют управлять линейными объектами с постоянными и переменными параметрами, а также управлять нелинейными объектами даже с учетом неидеальности управляющего устройства, а также осуществлять управление с применением скользящих режимов [9]. Также подобные теории могут основываться на вариациях метода линеаризации по обратной связи [19], однако все перечисленные методы в силу их большой общности не могут учитывать особенности и свойства механических систем, такие как законы сохранения или вид определяющих систему уравнений.
Тем не менее, был развит ряд методов, применимых к системам, обладающих механическими свойствами. К ним относятся методы приближенной декомпозиции на основе частичной линейной аппроксимации, усреднения и сингулярных возмущений, предписанное пространственное движение, согласованное управление, робастные алгоритмы адаптации высокого порядка, методы скоростного градиента и неявной эталонной модели [7].
Особое внимание следует уделить методу скоростного градиента [3], в том числе применяемому для постановки и решения классов задач, относящихся к возбуждению волн и колебаний заданной формы, скорости и другими свойствами. Алгоритм управления метода скоростного градиента работает за счет изменения управляющего воздействия по направлению градиента функции скорости изменения производной от целевой функции по времени вдоль траектории, определяемой уравнением управляемого объекта в пространстве состояний. 
В качестве классического примера управления этим методом может быть рассмотрена задача о раскачивании маятника [1]. Метод скоростного градиента также применим для задач с целевой функцией, выраженной через энергию системы, а также продемонстрировал свою эффективность в задачах адаптивного управления [10][11][12]. Именно поэтому данный метод может быть особенно полезен для управления нелинейными локализованными волнами в механических системах, так как может учитывать их главные в данном рассмотрении свойства.
Именно поэтому данный метод применялся для численных исследований управления, а именно генерации и локализации, а также поддержания свойств уединенных волн уравнения синус-Гордона в работе [22]. При определенных параметрах управления удавалось генерировать локализованные волны, не являющиеся точными решениями синус-Гордона, в том числе модифицированные кинки и колоколообразные солитоны. Также применимость метода оценивалась для обобщенных версий уравнения синус-Гордона – для двойного уравнения синус-Гордона и его же с дополнительным дисперсионным слагаемым. Исследования показали, что управление успешно работает и на обобщенных формах уравнения, при этом средствами управления численно удалось смоделировать взаимодействие движущихся навстречу друг другу солитонов колоколообразной формы.
Этот же метод применялся в работе [25] для управления волновыми процессами в связанных уравнениях. При этом управляющая функция вводилась лишь в одно уравнение, численно удалось показать, что алгоритм управления работает, не вызывая осцилляций в решении лишь при моментальном его включении сразу после генерации начального импульса. Таким образом применение данного метода было обобщен на управление волновыми процессами в более сложной системе и определены границы применимости этого обобщения.
Однако существенный недостаток метода скоростного градиента применительно к такого рода задачам заключается в том, что управление, вообще говоря, получается распределенным, и реализовать его на практике для конкретной физической системы зачастую бывает невозможно. В работе [23] была предложена реальная механическая система, в которой может осуществляться управление слабо-поперечными локализованными волнами упругости – упругий изотропный слой с особого типа нелинейностью, одна граница которого погружена в морозный грунт, а другая контактирует с распределенной нормальной нагрузкой, которая определяет воздействие на систему. При этом был предсказан новый алгоритм управления слабо-продольными волнами средствами тангенциальной нагрузки, а в работе [24] он был численно подтвержден на примере генерации волн сжатия-растяжения и бризерах разной формы и скорости. И все-таки, несмотря на приведенный в работе пример, более реалистичным и практически реализуемым является управление границей, внешним полем, либо другим параметром физической системы.
Похожая задача управления была рассмотрена ранее в работе [4]. Была исследована возможность контроля локализации кинк-антикинк решения нелинейного уравнения синус-Гордона, однако без применения метода скоростного градиента. Вместо этого было проведено численное исследование эволюции антикинка при произвольных начальных условиях и найден управляющий параметр уравнения, позволяющий подавить осцилляции на фронте волны и воздействовать на ее скорость, если в начальном условии скорость кинка не соответствует точному решению уравнения. Однако далеко не всегда существует физический смысл некого воздействия на один из параметров уравнения, поэтому для осуществления на практике предложенного алгоритма требуется соответствующая физическая система с возможностью воздействия на этот параметр.
В работе [18] был применен модифицированный алгоритм скоростного градиента для управления энергией волн в уравнении синус-Гордона методом воздействия на границу системы. Было численно продемонстрировано, что за конечное время систему, поведение которой описывается этим уравнением, можно привести к состоянию с любым ненулевым значением энергии. При этом численные результаты показали существенно малые оценки ошибок управления как при увеличении, так и при уменьшении энергии системы. Однако нужно подчеркнуть, что речь идет о косвенном управлении волнами, так как в данной задаче цель управления – определенное значение энергии. То есть управлять конкретной локализованной волной, а именно генерировать ее с определенной формой и скоростью, а также точно менять эти параметры при распространении волны данным алгоритмом не получится.
Следует вернуться к рассмотрению других методов управления нелинейными механическими системами, отличных от метода скоростного градиента, его модификаций, и которые отходят от обобщений и акцентируют свое внимание на механическую природу уравнений движения и за счет этого оказываются на порядок более применимыми в практическом смысле.
В работах [13] и [14] предложен один из таких методов управления, основанный на игровом подходе для случаев отсутствия сопротивления в системе и линейного сопротивления соответственно, который гарантирует приведение рассматриваемой системы к цели управления за конечное время. При этом в работе [15] данный метод был обобщен для случая любого вида нелинейного сопротивления. Также предложенный метод обладает малой чувствительностью к возмущениям, что является несомненным плюсом с точки зрения практической значимости.
В работе [8] рассматриваются задачи управления механическими системами в условиях структурной, силовой и информационной неопределенности. В сформулированной задаче речь идет об управлении не одной механической системой, а совокупностью систем, которая характеризуется набором определяющих параметров. Предложенный подход определяется как принцип декомпозиции в управлении механическими системами.
Невозможно уделить внимание абсолютно всем методам управления самыми различными системами, поэтому в данной работе мы ограничимся перечисленными. К тому же, большая часть оставшихся нерассмотренных подходов к управлению нелинейными системами не имеет ни прямого, ни косвенного отношения к волновым и колебательным процессам.
Стоит обратить внимание и на вопросы применимости разрабатываемых математических аппаратов и алгоритмов управления в других физических системах (не классических механических), так как многие методы опираются лишь на свойства уравнений, а не на свойства описываемых систем. Когда речь заходит о задачах управления в физических системах, сложно обойти стороной растущую популярность науки о применении метаматериалов. Их теория основывается на так называемом явлении локального резонанса, который используется для контроля над распространением электромагнитных, акустических и упругих волн в искусственно созданных средах. 
Первоначально внимание было сосредоточено на вопросе о возможности существования запрещенной зоны в материале, ширина которой меньше длины волны, и которая генерируется резонаторами. Существование такой зоны приводит к зависимости эффективных параметров созданной среды (метаматериала) от частоты, при этом среда при некоторых значениях частоты резонатора может обладать отрицательным показателем преломления. Сейчас исследования переходят к получению наиболее общих методов управления формами волн для получения среды с показателем преломления, меняющимся в пространстве по нужному закону. В области фотоники и акустики этот переход уже состоялся, и новые метаматериалы позволяют осуществлять управление распространением света, волнами микроволнового диапазона, волнами на поверхности воды, а также акустическими волнами. 
Например, один из уже существующих метаматериалов позволяет управлять акустическими волнами с целью удержания шарообразного тела в воздухе. Проходящие через него звуковые волны сгущаются в одну точку в пространстве подобно фокусировке электромагнитных волн, проходящих через линзу. В дальнейшем планируется создание метаматериала, позволяющего управлять внешними звуковыми волнами генерируемыми акустическими.
К вопросам управления нелинейными локализованными волнами также может обратиться и другая область науки – волоконная оптика. Оптическое волокно – нить из оптически прозрачного материала, используемая для переноса электромагнитных волн внутри себя посредством полного внутреннего отражения. В данной сфере солитоны обнаруживаются при самофокусировке – явление, при котором поперечное распределение электрического поля пучка света не меняется вдоль оси пучка. Оно описывается солитонными решениями нелинейного уравнения Шредингера. 
В области практической реализации, в целях передачи информации наибольшее распространение в настоящее время получают совмещенные оптоэлектронные и электрооптические схемы, представляющие собой дискретные оптические элементы, комбинированные со стандартными электронными узлами. Такое сочетание, как правило, не является принципиальным и обусловлено, в основном, отсутствием на современном этапе ряда оптических устройств, которые могли бы составить конкуренцию электронным аналогам. Трудности практической реализации названных устройств заключаются в том, что в основе их функционирования должны лежать операции управления электромагнитных волн светом. Такое управление возможно только на основе нелинейных взаимодействий волн в нелинейных средах. Поскольку в оптике коэффициенты нелинейной поляризуемости известных веществ крайне малы, то для получения заметных нелинейных эффектов необходимо обеспечивать очень большие плотности мощности.
Таким образом, исследование и разработка методов управления нелинейными волновыми процессами, очевидно, являются существенно востребованными в самых разных областях науки и техники в настоящее время. Наиболее подходящим и целесообразным способом управления локализованными волнами остаются алгоритмы, берущие в основу метод скоростного градиента и его модификации. Помимо классических механических систем следует обратить внимание на другие физические системы  
Заключение

В ходе проделанной работы были успешно изучены вопросы истории и методологии исследования нелинейных волновых процессов, было уделено необходимое внимание истории развития методов управления линейными и нелинейными механическими системами, при этом были подробно рассмотрены вопросы о применимости разрабатываемого математического аппарата в ходе исследовательской работы, с акцентом на методы исследования тех областей науки и техники, к которым он будет применяться в перспективе.
Было получено полноценное представление об истории развития науки о нелинейных волнах, рассмотрены трудности, с которыми сталкивались ученые при построении новых теорий, уделено необходимое внимание появившимся в ходе этого развития новых методов исследования. В рамках данной работы также были изучены современные подходы к управлению нелинейными системами, что несомненно будет полезно для дальнейшей исследовательской работы. Таким образом, все поставленные цели работы были успешно выполнены.


\fixme{Abstract оригинальной работы, заменить кратким описанием задачи, связать с основной работой каким-нибудь образом. В диссертации не выводятся модельные уравнения, а используются.}

В работе исследуется динамика механических структур с двумерной решеткой. Модель включает в себя взаимодействия между соседними и не соседними частицами решетки, учет как трансляционного, так и вращательного взаимодействий, а также их нелинейный характер (физическая нелинейность) взаимодействия. \fixme{Разработаны асимптотические процедуры для получения определяющих нелинейных уравнений движения в континуальном пределе}. Полученные уравнения исследуются как аналитически, так и численно. Особый интерес представляют распространение и поперечная неустойчивость плоских уединенных волн деформации. Показано, как отличается динамика продольных и поперечных волн разных двумерных решетках. Получены соотношения для упругих постоянных, характеризующие тип локализованных волн деформации (растяжение или сжатие), их поперечную неустойчивость и возможное ауксетическое поведение \fixme{$<-$ убрать}. Получены численные решения, описывающие неустойчивую и устойчивую динамику плоских продольных и поперечных волн.

Литература

[1] Акуленко JI. Д. Параметрическое управление колебаниями и вращением физического маятника (качелей) / / Прикл. математика и механика. 1993. Т. 57. N° 2. С 88-90

[2] Алексеев В. М., Тихомиров В. М., Фомин С. В. Оптимальное управление. - М.: Наука, 1979, УДК 519.6, - 223 c.

[3] Андриевский Б. Р., Гузенко П. Ю., Фрадков А. Л., “Управление нелинейными колебаниями механических систем методом скоростного градиента”, Автомат. и телемех., 1996, № 4, 4–17; Autom. Remote Control, 57:4 (1996), 456–467

[4] Бондаренков Р.С., Порубов А.В.. Управление локализацией нелинейных волн уравнения синус-Гордона. Процессы управления и устойчивость. 2015 Т.2. N°.1, с 29-33

[5] Браун О. М., Кившарь Ю. С. Модель Френкеля-Конторовой: Концепции, методы, приложения. 2008. Vol. 519 с.

[6]	Емельянов С. В. Системы автоматического управления с переменной структурой. — М.: Наука, 1967.

[7]	Мирошник И. В., Никифоров В. О., Фрадков А. Л. Нелинейное и адаптивное управления сложными динамическими системами. — С.-Пб.: Наука, 2000.

[8]	Пятницкий Е. С. Принцип декомпозиции в управлении механическими системами // ДАН СССР. - 1988.- Т. 300, № 2.- С.

[9]	Уткин В. П., Орлов Ю. В. Теория бесконечномерных систем управления на скользящих режимах. — М.: Наука, 1990.

[10] Фрадков A. Л. Схема скоростного градиента и ее применение в адаптивном управлении / / АиТ. 1979. №. 9. С. 90-101.

[11] Фомин В. H. Фрадков A. Л., Якубович В. А. Адаптивное управление динамическими объектами. М.: Наука, 1981.

[12] Фрадков А. Л. Адаптивное управление в сложных системах. М.: Наука, 1990

[13] Черноусько Ф. Л. Декомпозиция и субоптимальное управление в динамических системах // Прикл. матем. и мех. — 1990.- Т. 54, вып. 6.- С. 883- 893.

[14] Черноусько Ф. Л. Декомпозиция и синтез управления в динамических системах // Изв. АН СССР. Техн. кибернет. - 1990.- № 6.- С. 64-82.

[15] Черноусько Ф. Л. Синтез управления системой с нелинейным сопротивлением // Прикл. матем. и мех. — 1991.- Т. 55, вып. 6.- С.

[16] Airy G. B., Tides and waves. Encyclopaedia Metropolitana, 1845

[17] Boussinesq JV. Th´eorie de l’intumescence liquide appel´ee onde solitaire ou de translation, se propageant dans un canal rectangulaire. C. R. Acad. Sci. Paris 2:755–59, 1871

[18] Dolgopolik, Maksim and L. Fradkov, Alexander and Andrievsky, Boris. (2016). Boundary Energy Control of the Sine-Gordon Equation**This work was performed in IPME RAS, supported by RSF (grant 14-29-00142).. IFAC-PapersOnLine. 49. 148-153. 10.1016/j.ifacol.2016.07.1000.

[19]	Isidori A. Nonlinear Control Systems. — 3rd ed. — New York: Springer Verlag, 1995.

[20] Korteweg, D. J. and de Vries, G. On the Change of Form of Long Waves advancing in a Rectangular Canal and on a New Type of Long Stationary Wave // Philosophical Magazine. — 1895. — Vol. 39. — P. 422—443.

[21] Marsden Jerrold E. Mathematical foundations of elasticity, 1983 // Linearization

[22] Porubov A.V., Antonov I.D., Fradkov A.L., Andrievsky B.R.. Control of localized non-linear strain waves in complex crystalline lattices, International Journal of Non-Linear Mechanics 86 (2016) 174–184

[23] Porubov A.V., Antonov I.D., Indeitsev D.A., Fradkov A.L.. Mechanical system allowing distributive control with feedback, Mechanics Research Communications MRC-3199; No. of Pages 4

[24] Porubov A.V., Antonov I.D., Fradkov A.L.. Further progress in control of localized nonlinear waves, J. Phys.: Conf. Ser. 937 012043

[25] Porubov A.V., Antonov I.D.. Control of coupled localized nonlinear wave solutions. IOP Conf. Series: Journal of Physics: Conf. Series 788 (2017) 012029

[26] Rayleigh B. (J.W. Strutt). On waves. Philos. Mag. (5) 1:257–79, 1876

[27] Russell, J.S., Report on Waves. 14th Meeting of the British Association for the Advancement of Science, 1844

[28] Stokes, G. On the theory of oscillatory waves, from Transactions of the Cambridge Philosophical Society, vol. 8, p. 441, 1847

[29] Zabuski N. J., Kruskal M.D., Interaction of “Solitons” in a collisionless plasma and the recurrence of initial states, Phys. Rev. Lett., 15, 240243, 1965
