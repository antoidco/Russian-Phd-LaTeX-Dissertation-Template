%% Согласно ГОСТ Р 7.0.11-2011:
%% 5.3.3 В заключении диссертации излагают итоги выполненного исследования, рекомендации, перспективы дальнейшей разработки темы.
%% 9.2.3 В заключении автореферата диссертации излагают итоги данного исследования, рекомендации и перспективы дальнейшей разработки темы.

%\fixme{Из реферата по философии науки... Нужно исправить и дополнить.}

%В ходе проделанной работы были успешно изучены вопросы истории и методологии исследования нелинейных волновых процессов, было уделено необходимое внимание истории развития методов управления линейными и нелинейными механическими системами, при этом были подробно рассмотрены вопросы о применимости разрабатываемого математического аппарата в ходе исследовательской работы, с акцентом на методы исследования тех областей науки и техники, к которым он будет применяться в перспективе.
%Было получено полноценное представление об истории развития науки о нелинейных волнах, рассмотрены трудности, с которыми сталкивались ученые при построении новых теорий, уделено необходимое внимание появившимся в ходе этого развития новых методов исследования. В рамках данной работы также были изучены современные подходы к управлению нелинейными системами, что несомненно будет полезно для дальнейшей исследовательской работы. Таким образом, все поставленные цели работы были успешно выполнены.

\begin{enumerate}
  \item На основе анализа существующей литературы по тематике исследования были выбраны и сформулированы основные цели работы, поставлены задачи для их решения, обоснована их актуальность, практическая значимость, а также дальнейшие перспективы для развития.
  \item Разработаны модельные уравнения движения для слоя с заданными граничными условиями и видом нелинейной функции потенциальной энергии, исследована возможность управления волновыми процессами в такой механической системе по схеме скоростного градиента и определены границы применимости. 
  \item Разработан спектральный метод решения обобщенного уравнения Кадомцева-Петвиашвили с дополнительным интегральным слагаемым, описывающего распространение волн в двумерных обобщенных решетках. С помощью полученного метода численно продемонстрированы свойства устойчивости плоских волн разных типов, предсказанные аналитически. 
  \item Численно исследованы свойства распространения и генерации гармонических волн для модели метаматериала <<масса-в-массе>> в условиях наличия запрещенной зоны при различных параметрах системы. 
\end{enumerate}
