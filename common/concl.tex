%% Согласно ГОСТ Р 7.0.11-2011:
%% 5.3.3 В заключении диссертации излагают итоги выполненного исследования, рекомендации, перспективы дальнейшей разработки темы.
%% 9.2.3 В заключении автореферата диссертации излагают итоги данного исследования, рекомендации и перспективы дальнейшей разработки темы.

\fixme{Из реферата по философии науки... Нужно исправить и дополнить.}

В ходе проделанной работы были успешно изучены вопросы истории и методологии исследования нелинейных волновых процессов, было уделено необходимое внимание истории развития методов управления линейными и нелинейными механическими системами, при этом были подробно рассмотрены вопросы о применимости разрабатываемого математического аппарата в ходе исследовательской работы, с акцентом на методы исследования тех областей науки и техники, к которым он будет применяться в перспективе.
Было получено полноценное представление об истории развития науки о нелинейных волнах, рассмотрены трудности, с которыми сталкивались ученые при построении новых теорий, уделено необходимое внимание появившимся в ходе этого развития новых методов исследования. В рамках данной работы также были изучены современные подходы к управлению нелинейными системами, что несомненно будет полезно для дальнейшей исследовательской работы. Таким образом, все поставленные цели работы были успешно выполнены.

\begin{enumerate}
  \item На основе анализа \ldots
  \item Численные исследования показали, что \ldots
  \item Математическое моделирование показало \ldots
  \item Для выполнения поставленных задач был создан \ldots
\end{enumerate}
